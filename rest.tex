%LTeX: language=pt-BR
%!TEX root = principal.tex
\chapter{Descrição e avaliação do apoio institucional recebido no período}\label{chp:apoioInst}
O Departamento de Matemática Aplicada do Instituto de Matemática, Estatística e Computação Científica (IMECC) contribuiu com o projeto através do acesso ao LOC (Laboratório de Otimização Contínua) e a seus computadores de alto desempenho, essenciais para os experimentos numéricos realizados. Além disso, o SBU (Sistema de Bibliotecas da Unicamp) forneceu acesso ao acervo bibliográfico e aos periódicos assinados pela UNICAMP e CAPES, fundamentais para a revisão bibliográfica.

\chapter{Participação em evento científico}\label{chp:particEvento}
O aluno participou do 1\textsuperscript{st} Carioca Workshop on Optimization and Applications (CariOPT 2025), realizado na Escola de Matemática Aplicada da Fundação Getúlio Vargas (FGV EMAp), no Rio de Janeiro, de 05 a 07 de maio de 2025.

Na ocasião, foi apresentado o trabalho intitulado \textit{``Optimization of problems involving group sparsity''}, divulgando os resultados preliminares e a temática central da pesquisa.

\chapter{Plano de atividades para o próximo período}\label{chp:planoAtiv}
Para o próximo período de vigência (03/2026 a 02/2027), o projeto prevê as seguintes etapas principais:

\begin{enumerate}
    \item Atividades Acadêmicas:
    Cumprimento dos créditos restantes e preparação para o \textit{Exame de Qualificação Específico} na disciplina MT601 - Métodos Computacionais de Otimização, conforme exigência do programa de pós-graduação para a linha de pesquisa;

    \item Redação e Submissão de Artigo Científico: 
    Organização dos resultados obtidos, em particular a comparação entre o método proposto (NSPG) e o estado da arte (L0Learn), bem como a análise das estratégias híbridas, visando a submissão de um artigo para um periódico de circulação internacional na área de otimização;

    \item Aprofundamento dos Experimentos Numéricos: 
    Expandir a bateria de testes para incluir:
    \begin{itemize}
        \item Problemas de maior escala para avaliar a escalabilidade das implementações;
        \item Aplicações com dados reais (e.g., genômica ou processamento de imagens) para validar a eficácia prática da regularização $\ell_0$ nos modelos propostos;
        \item Testes de robustez em cenários mal-condicionados e com ruído elevado;
    \end{itemize}

    \item Participação em Evento Internacional:
    Participação confirmada em um Minisimpósio na \textit{SIAM Conference on Optimization (OP26)}, que ocorrerá de 02 a 05 de junho de 2026, na University of Edinburgh, Edimburgo, Reino Unido;

    \item Estágio de Pesquisa no Exterior (BEPE):
    Preparação e submissão de proposta para realização de Estágio de Pesquisa no Exterior (BEPE) na \textit{Polytechnique Montr\'eal}, Canadá, com início previsto para setembro de 2026. O projeto será supervisionado pelo Prof. Dr. Dominique Orban, Professor Titular do Departamento de Matemática e Engenharia Industrial e pesquisador afiliado ao GERAD (Group for Research in Decision Analysis) e IVADO (Institute for Data Valorization), referência internacional em otimização numérica e computacional. As tratativas preliminares com o pesquisador já estão em andamento;

    \item Investigação Teórica:
    Estudar a convergência global das estratégias híbridas desenvolvidas, buscando estabelecer garantias teóricas para a combinação de métodos de descida coordenada com inicializações espectrais não-monótonas;
\end{enumerate}

%\chapter{Lista das publicações resultantes do auxílio no período a que se refere o Relatório Científico}\label{chp:publicacoes}

%\chapter{Lista dos trabalhos preparados ou submetidos}\label{chp:listaPrepar}

%%-----
%% Referências bibliográficas
%%-----
\addcontentsline{toc}{chapter}{\bibname}
\bibliographystyle{abbrv} 
\bibliography{bibliografia}