%LTeX: language=pt-BR
%%%%%%%%%%%%%%%%%%%%%%%%%%%%%%%%%%%%%%%%%%%%%%%%%%%%%%%%%%%%%%%%%%%%%
% In English:
%    This is a Latex template for São Paulo Research Foudation (FAPESP)
%         reports (annual or final).
%    This is the modified version of the original Latex template from
%         following website.
%    Original Source: http://www.howtotex.com
%    For information about FAPESP, check http://www.fapesp.br/en
%    This template targets mainly on reports in Portuguese language.
%    New additions and changes in the latest version:
%        - Added the possibility of including multiple members in the 
%          research team, with the commands \memberA{Name of Member A} 
%          \memberB{Name of Member B} \memberC{Name of Member C} etc.
%        - Included commands to define project modality and the research 
%          agency (if you want to use the same model for other research 
%          agencies such as CAPES, CNPq etc).
%
% In Portuguese:
%    Este é um modelo Latex para relatórios (anual ou final) da Fundação 
%         de Amparo à pesquisa do Estado de São Paulo (FAPESP).
%    Esta é uma versão modificada do modelo Latex do site supra mencionado.
%    Para informações sobre a FAPESP, verifique http://www.fapesp.br
%    Esse modelo foca principalmente nos relatórios escritos em Português.
%    Novas adições e alterações na última versão:
%       - Foi adicionada a possibilidade de incluir vários membros no 
%         grupo de pesquisas, com os comandos \membroA{Nome do Membro A} 
%         \membroB{} \membroC{} etc.
%       - Foram incluídos comandos para definir modalidade de projeto e
%         agência de fomento (caso queira utilizar o mesmo modelo para 
%         outras agências, CAPES, CNPq etc).
%
% Author/Autor: André Leon Sampaio Gradvohl, Dr.
% Email:        andre.gradvohl@gmail.com
% Lattes CV:    http://lattes.cnpq.br/9343261628675642
% GitHub: http://gradvohl.github.io/
% 
% Last update/Última versão: 19/Feb/2018
%
%%%%%%%%%%%%%%%%%%%%%%%%%%%%%%%%%%%%%%%%%%%%%%%%%%%%%%%%%%%%%%%%%%%%%%
\documentclass[12pt]{report}
\usepackage[a4paper]{geometry}
\usepackage[english,portuguese]{babel}
\usepackage[myheadings]{fullpage}
\usepackage[T1]{fontenc}
\usepackage{fancyhdr}
\usepackage{graphicx}
\usepackage{setspace}
\usepackage{sectsty}
%\usepackage{url}

%\PassOptionsToPackage{dvipsnames}{xcolor}
%\usepackage{listingsutf8}
\usepackage{mathptmx}
\DeclareMathAlphabet{\mathcal}{OMS}{cmsy}{m}{n}
\usepackage{amsthm}
\usepackage{microtype}
\usepackage{listings}
\usepackage[shortlabels]{enumitem}
\usepackage{amsmath}
%\usepackage{comment}
\usepackage{index}
\usepackage{makecell}
\usepackage{wrapfig}
%\numberwithin{equation}{section}
\usepackage{float}
\usepackage{hyperref} %\usepackage[hyphens]{url}
\usepackage{algorithm}
\usepackage[table]{xcolor}
\usepackage{authblk}
\usepackage{algpseudocode}
\usepackage{tikz}
\usepackage{amssymb}
\usepackage{nicematrix}
\usepackage{mathtools}
\usepackage{xcolor}
\usepackage{subfig}
\usepackage{soul}
\usepackage{thmtools}
\usepackage{bm}
%\usepackage{pstricks,pstricks-add}
\usepackage{tabto}
\usetikzlibrary{patterns,decorations.pathreplacing}
\graphicspath{{images/}}

\hypersetup{linkcolor=black}

\newtheorem{theorem}{Teorema}%[chapter]
\newtheorem{lemma}{Lema}%[chapter]
\providecommand*{\lemmaautorefname}{Lema}
\newtheorem{assumption}{Hipótese}%[chapter]
\providecommand*{\assumptionautorefname}{Hipótese}
\newtheorem{proposition}{Proposição}%[chapter]
\providecommand*{\propositionautorefname}{Proposição}
\newtheorem{corollary}{Corolário}%[chapter]
\providecommand*{\corollaryautorefname}{Corolário}
\newtheorem{conjecture}{Conjectura}%[chapter]
\providecommand*{\conjectureautorefname}{Conjectura}
\newtheorem{definition}{Definição}%[chapter]
\providecommand*{\definitionautorefname}{Definição}
\newtheorem{notation}{Notação}%[chapter]
\providecommand*{\notationautorefname}{Notação}
\newtheorem{remark}{Observação}%[chapter]
\providecommand*{\remarkautorefname}{Observação}
\newtheorem{exmp}{Exemplo}%[chapter]
\providecommand*{\exampleautorefname}{Exemplo}
\newtheorem{note}{Nota}%[chapter]
\providecommand*{\noteautorefname}{Nota}
\providecommand*{\algorithmautorefname}{Algoritmo}
\addto\extrasbrazil{
    \def\sectionautorefname{Seção}
}
\addto\extrasbrazil{
    \def\subsectionautorefname{Subseção}
}
\renewcommand*{\chapterautorefname}{Capítulo}
\renewcommand*{\partautorefname}{Parte}
\renewcommand*{\figureautorefname}{Figura}

\newcommand*\R{\mathbb{R}}
\newcommand*\X{\mathbb{X}}
\newcommand{\hquad}{\mkern3mu}
\newcommand*\C{\mathbb{C}}
\newcommand*\N{\mathbb{N}}
\newcommand*\K{\mathbb{K}}
\newcommand*\x{\mathbf{x}}
\newcommand*\D{\mathbf{d}}
\newcommand*\bv{\mathbf{b}}
\newcommand*\U{\mathbf{u}}
\newcommand*\y{\mathbf{y}}
\newcommand*\z{\mathbf{z}}
\newcommand*\V{\mathbf{v}}
\newcommand*\s{\mathbf{s}}
\newcommand*\A{\mathbf{A}}
\newcommand*\prox{\operatorname{prox}}
\newcommand{\sumover}[2]{\sum \limits_{#1}^{#2}}
%\renewcommand{\qedsymbol}{{$\blacksquare$}}
\newcommand*{\norm}[1]{\| #1 \|}
\newcommand*{\inner}[1]{\langle #1 \rangle}
\newcommand*{\modu}[1]{| #1 |}
\newcommand{\vecspace}[3][]{(#2,\norm{#3}_{#1})}

\algnewcommand{\Input}[1]{\textbf{Input:} #1}
\algnewcommand{\Output}[1]{\textbf{Output:} #1}
\newcommand\algoIndent[1]{%
  \linebreak\hspace*{\dimexpr\algorithmicindent*#1}\hspace*{-0.33em}%
}
%\algdef{SE}[DOWHILE]{Do}{doWhile}{\algorithmicdo}{\algorithmicwhile\ }%
%%------ 
%% Comandos gerais
%% Observação: o arquivo "comandos.tex" tem que estar presente.
%%------


\providecommand*{\subfigureautorefname}{Subfigura}

\renewcommand*{\listalgorithmname}{Lista de algoritmos}
\input{comandos}
%
%%-----
%% Página de título
%% Observação: As definições que aparecem a seguir comporão a
%%             página de título e a folha de rosto.
%%-----
%% Define o nome da universidade onde o projeto foi desenvolvido.
\universidade{Universidade Estadual de Campinas}
%
%% Define o nome da faculdade onde o projeto foi desenvolvido.
\faculdade{Departamento de Matemática Aplicada, Instituto de Matemática, Estatística e Computação Científica}
%
%
%% Define o título do projeto.
\titulo{Métodos de segunda ordem para problemas compósitos descontínuos}
%
%% Define a agencia de Fomento e a abreviatura. O primeiro argumento é o 
%% nome por extenso e o segundo a abreviatura.
%% Ambos os argumentos são obrigatórios
\agFomento{Fundação de Amparo à Pesquisa do Estado de São Paulo}{FAPESP}
%
%% Define o tipo de relatório. Pode ser Anual ou Final.
%% Não é obrigatório definir o tipo de relatório.
\tipoRelatorio{Primeiro Relatório Científico Anual}
%
%% Define a modalidade de Projeto. Pode ser temático, regular, etc.
\modalidadeProjeto{Bolsa no País - Doutorado}
%
%% Define o número do projeto.
%% Não é obrigatório definir o número do projeto.
%
%% Define o número do projeto.
%% Não é obrigatório definir o número do projeto.
\numProjeto{2024/20168-8} 
%
%% Define o autor do relatório.
\autor{Paulo José da Silva e Silva \\ Doutorando: Gabriel Belém Barbosa}
%
%% Define a equipe do projeto (incluindo o pesquisador responsável no comando \membroA{}
\membroA{Paulo José da Silva e Silva}
%% Inclua os demais membros do grupo (máximo +5)
\membroB{Gabriel Belém Barbosa}
%\membroC{Francisco}
%\membroD{Joao}
%\membroE{Antonio}
%\membroF{José}
%
%% Define o período da vigência do Projeto.
\periodoVigencia{01/03/2025 a 28/02/2026}
%
%% Define o período coberto pelo relatório.
\periodoRelatorio{01/03/2025 a 28/02/2029}
%
%% Define a cidade onde o projeto foi desenvolvido.
\cidade{Campinas}

%%-----
%% Página de título
%% Observação: Os comandos a seguir não devem ser mudados, 
%%             exceto caso necessário.
%%-----
\begin{document}
%
%% Define a numeração em romanos.
\pagenumbering{roman}
%
%% Gera a folha de título.
\geraTitulo
%
%% Gera a folha de rosto.
\folhaDeRosto
%
%% Escreva aqui o resumo em português.
\Resumo{
  Este projeto visa investigar e avançar no campo da otimização compósita não convexa, focando na recente incorporação de informação de segunda ordem nos modelos iterativos. Duas referências recentes se destacam nessa área: \textit{Forward-backward envelope for the sum of two nonconvex functions: further properties and nonmonotone line-search algorithms} [Themeles 2018] e \textit{The Indefinite Proximal Gradient Method} [Leconte 2024]. Vários problemas do mundo real, incluindo recuperação de sinal esparso, processamento de imagem e otimização de portfólio, podem ser modelados na forma do problema de interesse. Esse trabalho também envolverá extensa experimentação numérica em várias classes de problemas para avaliar o desempenho dos algoritmos propostos. Os experimentos se concentrarão em comparar o desempenho de diferentes estratégias de busca linear sob diferentes hipóteses, especialmente estratégias não monótonas, o impacto da incorporação de informações quasi-Newton nos modelos do subproblema e a eficácia de cada atualização. Possivelmente novas estratégias híbridas também serão analisadas. O projeto pretende avançar no entendimento de métodos como o ZeroFPR [Themeles 2018] e de gradiente proximal indefinido [Leconte 2024] no tratamento de problemas de otimização compósita não convexa, fornecer orientação prática na seleção de estratégias de busca linear adequadas e atualizações quasi-Newton para diferentes estruturas de problema e desenvolver algoritmos eficientes e robustos para resolver uma classe mais ampla de problemas de otimização compósita não convexa, levando potencialmente a soluções aprimoradas em importantes áreas de aplicação.
  }
%
%% Escreva aqui o resumo em inglês.
\Abstract{
 This project aims to investigate and advance the field of nonconvex composite optimization, focusing on the recent incorporation of second-order information into iterative models. Two recent sources stand out in this area: \textit{Forward-backward envelope for the sum of two nonconvex functions: further properties and nonmonotone line-search algorithms} [Themeles 2018] and \textit{The Indefinite Proximal Gradient Method} [Leconte 2024]. Several real-world problems, including sparse signal recovery, image processing, and portfolio optimization, can be modeled in the form of the problem of interest. This work will also involve extensive numerical experimentation on several types of problems to evaluate the performance of the proposed algorithms. The experiments will focus on comparing the performance of different line search strategies with or without relaxed assumptions, especially nonmonotone variants, the impact of incorporating quasi-Newton information into the subproblem and the effectiveness of each update. New hybrid strategies also possibly will be analysed. The project aims to contribute to advance the understanding of methods such as ZeroFPR [Themeles 2018] and indefinite proximal gradient [Leconte 2024] in the treatment of non-convex composite optimization problems, to provide practical guidance in the selection of appropriate line search strategies and quasi-Newton updates for different problem structures, and to develop efficient and robust algorithms to solve a broader class of non-convex composite optimization problems, potentially leading to improved solutions in important application areas.
}
%
%% Adicionará o sumário.
%% Mantenha o \thispagestyle{empty} e \clearpage
\tableofcontents
\thispagestyle{empty}
\clearpage
%
%% Define a numeração em arábicos.
\pagenumbering{arabic}

%%-----
%% Formatação do título da seção
%%-----
\sectionfont{\scshape}

%%-----
%% Corpo do texto
%%-----
% %LTeX: language=pt-BR
%!TEX root = principal.tex

\chapter{Resumo do projeto proposto}\label{chp:resumoProj} 
Este projeto visa investigar e avançar no campo da otimização compósita não convexa, focando na recente incorporação de informação de segunda ordem nos modelos iterativos. Duas referências recentes se destacam e servem de base para o desenvolvimento: \textit{Sparse regression at scale: branch-and-bound rooted in first-order optimization} \cite{mio} e \textit{First-Order Methods in Optimization} \cite{beckbook}. 

Vários problemas reais, incluindo recuperação de sinal esparso, processamento de imagem e otimização de portfólio, podem ser modelados na forma do problema de interesse. Esse trabalho também envolverá extensa experimentação numérica em várias classes de problemas para avaliar o desempenho dos algoritmos propostos. Os experimentos se concentrarão em comparar o desempenho de diferentes estratégias de otimização combinatorial local, o impacto da incorporação de informações de segunda ordem nos modelos do subproblema e a eficácia de cada atualização.

O projeto pretende avançar no entendimento de métodos híbridos que combinam otimização contínua e discreta \cite{mio} no tratamento de problemas de otimização compósita não convexa, fornecer orientação prática na seleção de estratégias adequadas para diferentes estruturas de problema e desenvolver algoritmos eficientes e robustos, levando potencialmente a soluções aprimoradas em importantes áreas de aplicação.


%LTeX: language=pt-BR
%!TEX root = principal.tex

\chapter{Realizações no período}\label{chp:realizacoes}

Durante a vigência deste primeiro período do projeto de doutorado (03/2025 a 02/2026), as atividades planejadas foram cumpridas, com destaque para a conclusão dos créditos obrigatórios, aprovação em exames de qualificação e participação em evento científico, além do andamento da pesquisa bibliográfica e experimental.

\section{Atividades Acadêmicas}

No âmbito das exigências do programa de Doutorado em Matemática Aplicada do IMECC/Unicamp, foram realizadas as seguintes atividades:

\begin{itemize}
    \item Coeficiente de Rendimento (CR): O aluno mantém um CR perfeito de 4,0000;
    \item Disciplinas Cursadas no Doutorado (1º Semestre de 2025):
    \begin{itemize}
        \item MT504 - Fluxos em Redes: Aprovado com conceito A;
        \item MT853 - Tópicos em Otimização: Aprovado com conceito A;
    \end{itemize}
    \item Aproveitamento de Créditos: As disciplinas obrigatórias MT401 - Análise Aplicada e MT402 - Matrizes foram aproveitadas do Mestrado (registradas como MT801 e MT802 no 2º semestre de 2025), cumprindo os requisitos de créditos obrigatórios;
    \item Total de Créditos: Foram totalizados 18 créditos no Doutorado (incluindo aproveitamentos), avançando significativamente rumo à integralização dos créditos exigidos pelo programa;
    \item Estágio Docente: Realização do Estágio de Capacitação Docente (PED C) na disciplina MS211 - Cálculo Numérico (código CD003), sob supervisão, durante o 2º semestre de 2025;
    \item Exame de Proficiência em Inglês: Aprovação no Exame de Proficiência em Inglês II (Escrito), realizado em 30/05/2025;
    \item Exames de Qualificação:
    \begin{itemize}
        \item Aprovação no Exame de Qualificação na área de Análise Aplicada (MT401), realizado em 25/08/2025;
        \item Aprovação no Exame de Qualificação na área de Matrizes (MT402), realizado em 27/08/2025.
    \end{itemize}
\end{itemize}

\section{Atividades de Pesquisa}

As atividades de pesquisa concentraram-se na investigação e aprimoramento de métodos de segunda ordem para otimização compósita não convexa, com ênfase no problema de regularização $\ell_0$:
\begin{equation}
    \min_{x \in \mathbb{R}^n} f(x) + \lambda \|x\|_0,
\end{equation}
onde $f(x)$ é uma função suave (perda quadrática ou logística) e $\lambda > 0$ é o parâmetro de regularização.

Os principais avanços incluem:

\subsection{Revisão Bibliográfica}
Estudo aprofundado das referências centrais e do estado da arte:
\begin{itemize}
    \item \textit{Sparse regression at scale: branch-and-bound rooted in first-order optimization} \cite{mio};
    \item \textit{First-Order Methods in Optimization} \cite{beckbook};
    \item \textit{Fast Best Subset Selection} \cite{fastselect}, para comparação com métodos de seleção de subconjuntos.
\end{itemize}

\section{Desenvolvimento Algorítmico e Experimental}
Foi realizada uma extensa campanha de experimentos e desenvolvimentos teóricos no ambiente \texttt{Julia}, focada na robustez e eficiência dos algoritmos para o problema de seleção de variáveis. As principais contribuições são detalhadas a seguir.

\subsection{Estratégias Avançadas de Validação Cruzada}
Implementamos e comparamos novas estratégias para a seleção do hiperparâmetro $\lambda$ ao longo do caminho de regularização, visando mitigar a convergência para mínimos locais ruins. Essa estratégia é particularmente essencial em contextos nos quais a esparsidade da verdade base (\textit{ground truth}) não é conhecida, como em problemas de seleção de variáveis em aprendizado de máquina.

A primeira abordagem desenvolvida, denominada \emph{CV Inverso}, inverte a lógica padrão de percorrer o caminho de regularização. Ao invés de iniciar com $\lambda_{high}$ (solução esparsa) e relaxar a penalidade, esta estratégia percorre o caminho de $\lambda_{low}$ a $\lambda_{high}$, iniciando de soluções mais densas após uma fase de \textit{warmup}. Observamos que iniciar com o suporte cheio e remover variáveis progressivamente evita o custo combinatório de adicionar variáveis uma a uma e é particularmente robusto para métodos espectrais como o NSPG.

Além disso, desenvolvemos o \emph{CV Adaptativo Inteligente}, uma abordagem híbrida que sonda os erros de validação nos dois extremos do intervalo de $\lambda$ e seleciona a melhor direção de varredura. O algoritmo permite ainda uma única reversão de direção caso detecte uma melhora significativa no erro de validação (maior que 1\%), permitindo explorar vales de mínimos locais que seriam ignorados por varreduras monotônicas rígidas. Esse critério também permite que o método termine mais rapidamente sem varrer por completo o \textit{range} de possíveis valores de $\lambda$, por vezes superando a padrão ou a inversa em velocidade.

\subsection{Integração Híbrida e Escalamento Dinâmico}
Para algoritmos baseados em passos espectrais, como o NSPG e VMNSPG, a escala do passo $\gamma_{k,0}$ varia dinamicamente. Para garantir a consistência de seleção do operador proximal, implementamos um ajuste no caminho de regularização onde os valores de teste de $\lambda$ são compensados pela magnitude do passo espectral corrente. Isso assegura que o operador proximal aplique o \textit{threshold} efetivo adequado, estabilizando a seleção do suporte.

Essa técnica é fundamental para a estratégia híbrida desenvolvida, que combina as forças de diferentes classes de algoritmos em dois estágios. Na fase global, utilizamos o Gradiente Proximal Espectral Não-Monótono (NSPG) para escapar de mínimos locais rasos e identificar rapidamente um suporte razoável. Na fase local subsequente, estratégias de \textit{Coordinate Partial Swap} (CPSI) garantem otimalidade combinatorial de ordem superior. Também testamos o uso das soluções do NSPG como ponto de partida para o \textit{Partially Greedy Cyclic Coordinate Descent} (PGCCD) \cite{fastselect}, gerando um método híbrido. O PGCCD converge rapidamente para um mínimo de alta precisão, refinando os coeficientes no suporte identificado, antes da busca por otimalidade combinatorial.

\subsection{Resultados Preliminares}
Para validar a eficácia dos métodos, utilizamos dados sintéticos gerados segundo o modelo linear $y = Ax + \epsilon$, onde $n$ denota o número de amostras, $p$ o número de parâmetros (variáveis) e $k^\dagger$ o tamanho do suporte da verdade terrestre (\textit{ground truth}). A matriz de dados $A$ possui colunas normalizadas e estrutura de correlação controlada pelo parâmetro $\rho$ (correlação entre variáveis adjacentes com decaimento exponencial ou correlação constante). O nível de ruído é determinado pelo \textit{Signal to Noise Ratio} (SNR), que varia de acordo com a dificuldade do cenário analisado.

A qualidade da solução recuperada é medida pela métrica de Similaridade do Suporte (Jaccard Modificado), definida como $J(S, S^\dagger) = \frac{|S \cap S^\dagger|}{|S \cup S^\dagger|}$, onde $S$ é o suporte estimado pelo algoritmo e $S^\dagger$ é o suporte verdadeiro. Esta métrica varia de 0 (disjunção total) a 1 (recuperação perfeita).

\begin{figure}[ht]
    \centering
    \begin{minipage}{0.48\textwidth}
        \centering
        \includegraphics[width=\linewidth]{images/mixed_sim.png}
        \caption{Similaridade do Suporte (Jaccard Modificado).}
        \label{fig:results_sim}
    \end{minipage}\hfill
    \begin{minipage}{0.48\textwidth}
        \centering
        \includegraphics[width=\linewidth]{images/mixed_time.png}
        \caption{Tempo de Execução (s).}
        \label{fig:results_time}
    \end{minipage}
    \caption{Comparação entre o método proposto (PGCCD/NSPG) e o estado da arte (L0Learn variants) em cenário de correlação exponencial ($\rho=0.5, p=2000, n=500, k^{\dagger}=100, \text{SNR}=10$). Os métodos propostos atingem recuperação de suporte superior ou equivalente com tempos competitivos.}
    \label{fig:results_comp_exp}
\end{figure}

Além do cenário exponencial, avaliamos o desempenho em configurações de correlação constante, que impõem desafios diferentes à estrutura de correlação das variáveis. A Figura \ref{fig:results_comp_const} apresenta os resultados para $\rho=0.9$ e $p=1000$.

\begin{figure}[ht]
    \centering
    \begin{minipage}{0.48\textwidth}
        \centering
        \includegraphics[width=\linewidth]{images/const_sim.png}
        \caption{Similaridade do Suporte (Jaccard Modificado).}
        \label{fig:results_sim_const}
    \end{minipage}\hfill
    \begin{minipage}{0.48\textwidth}
        \centering
        \includegraphics[width=\linewidth]{images/const_time.png}
        \caption{Tempo de Execução (s).}
        \label{fig:results_time_const}
    \end{minipage}
    \caption{Desempenho em cenário de correlação constante ($\rho=0.9, p=1000, n=250, k^{\dagger}=20, \text{SNR}=5$). Neste regime de alta correlação e ruído mais acentuado, os métodos propostos demonstram superioridade marcante na qualidade da solução recuperada em comparação com as alternativas, mantendo a eficiência computacional.}
    \label{fig:results_comp_const}
\end{figure}

Os resultados ilustrados nas Figuras \ref{fig:results_comp_exp} e \ref{fig:results_comp_const} indicam a robustez da combinação proposta. Em cenários desafiadores de ruído considerável e alta correlação entre os parâmetros — seja ela exponencial ou constante — a abordagem híbrida supera ou iguala o estado da arte (L0Learn) em termos de qualidade da solução recuperada e tempo de execução.


%LTeX: language=pt-BR
%!TEX root = principal.tex
\chapter{Descrição e avaliação do apoio institucional recebido no período}\label{chp:apoioInst}
O Departamento de Matemática Aplicada do Instituto de Matemática, Estatística e Computação Científica (IMECC) contribuiu com o projeto através do acesso ao LOC (Laboratório de Otimização Contínua) e a seus computadores de alto desempenho, essenciais para os experimentos numéricos realizados. Além disso, o SBU (Sistema de Bibliotecas da Unicamp) forneceu acesso ao acervo bibliográfico e aos periódicos assinados pela UNICAMP e CAPES, fundamentais para a revisão bibliográfica.

\chapter{Participação em evento científico}\label{chp:particEvento}
O aluno participou do 1\textsuperscript{st} Carioca Workshop on Optimization and Applications (CariOPT 2025), realizado na Escola de Matemática Aplicada da Fundação Getúlio Vargas (FGV EMAp), no Rio de Janeiro, de 05 a 07 de maio de 2025.

Na ocasião, foi apresentado o trabalho intitulado \textit{``Optimization of problems involving group sparsity''}, divulgando os resultados preliminares e a temática central da pesquisa.

\chapter{Plano de atividades para o próximo período}\label{chp:planoAtiv}
Para o próximo período de vigência (03/2026 a 02/2027), o projeto prevê as seguintes etapas principais:

\begin{enumerate}
    \item Atividades Acadêmicas:
    Cumprimento dos créditos restantes e preparação para o \textit{Exame de Qualificação Específico} na disciplina MT601 - Métodos Computacionais de Otimização, conforme exigência do programa de pós-graduação para a linha de pesquisa;

    \item Redação e Submissão de Artigo Científico: 
    Organização dos resultados obtidos, em particular a comparação entre o método proposto (NSPG) e o estado da arte (L0Learn), bem como a análise das estratégias híbridas, visando a submissão de um artigo para um periódico de circulação internacional na área de otimização;

    \item Aprofundamento dos Experimentos Numéricos: 
    Expandir a bateria de testes para incluir:
    \begin{itemize}
        \item Problemas de maior escala para avaliar a escalabilidade das implementações;
        \item Aplicações com dados reais (e.g., genômica ou processamento de imagens) para validar a eficácia prática da regularização $\ell_0$ nos modelos propostos;
        \item Testes de robustez em cenários mal-condicionados e com ruído elevado;
    \end{itemize}

    \item Participação em Evento Internacional:
    Participação confirmada em um Minisimpósio na \textit{SIAM Conference on Optimization (OP26)}, que ocorrerá de 02 a 05 de junho de 2026, na University of Edinburgh, Edimburgo, Reino Unido;

    \item Estágio de Pesquisa no Exterior (BEPE):
    Preparação e submissão de proposta para realização de Estágio de Pesquisa no Exterior (BEPE) na \textit{Polytechnique Montr\'eal}, Canadá, com início previsto para setembro de 2026. O projeto será supervisionado pelo Prof. Dr. Dominique Orban, Professor Titular do Departamento de Matemática e Engenharia Industrial e pesquisador afiliado ao GERAD (Group for Research in Decision Analysis) e IVADO (Institute for Data Valorization), referência internacional em otimização numérica e computacional. As tratativas preliminares com o pesquisador já estão em andamento;

    \item Investigação Teórica:
    Estudar a convergência global das estratégias híbridas desenvolvidas, buscando estabelecer garantias teóricas para a combinação de métodos de descida coordenada com inicializações espectrais não-monótonas;
\end{enumerate}

%\chapter{Lista das publicações resultantes do auxílio no período a que se refere o Relatório Científico}\label{chp:publicacoes}

%\chapter{Lista dos trabalhos preparados ou submetidos}\label{chp:listaPrepar}

%%-----
%% Referências bibliográficas
%%-----
\addcontentsline{toc}{chapter}{\bibname}
\bibliographystyle{abbrv} 
\bibliography{bibliografia}

%%-----
%% Fim do documento
%%-----
\end{document}