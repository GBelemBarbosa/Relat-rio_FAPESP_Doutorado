%LTeX: language=pt-BR
%!TEX root = principal.tex

\chapter{Resumo do projeto proposto}\label{chp:resumoProj} 
Este projeto visa investigar e avançar no campo da otimização compósita não convexa, focando na recente incorporação de informação de segunda ordem nos modelos iterativos. Duas referências recentes se destacam e servem de base para o desenvolvimento: \textit{Sparse regression at scale: branch-and-bound rooted in first-order optimization} \cite{mio} e \textit{First-Order Methods in Optimization} \cite{beckbook}. 

Vários problemas reais, incluindo recuperação de sinal esparso, processamento de imagem e otimização de portfólio, podem ser modelados na forma do problema de interesse. Esse trabalho também envolverá extensa experimentação numérica em várias classes de problemas para avaliar o desempenho dos algoritmos propostos. Os experimentos se concentrarão em comparar o desempenho de diferentes estratégias de otimização combinatorial local, o impacto da incorporação de informações de segunda ordem nos modelos do subproblema e a eficácia de cada atualização.

O projeto pretende avançar no entendimento de métodos híbridos que combinam otimização contínua e discreta \cite{mio} no tratamento de problemas de otimização compósita não convexa, fornecer orientação prática na seleção de estratégias adequadas para diferentes estruturas de problema e desenvolver algoritmos eficientes e robustos, levando potencialmente a soluções aprimoradas em importantes áreas de aplicação.
